\documentclass[12pt,]{article}
\usepackage{lmodern}
\usepackage{amssymb,amsmath}
\usepackage{ifxetex,ifluatex}
\usepackage{fixltx2e} % provides \textsubscript
\ifnum 0\ifxetex 1\fi\ifluatex 1\fi=0 % if pdftex
  \usepackage[T1]{fontenc}
  \usepackage[utf8]{inputenc}
\else % if luatex or xelatex
  \ifxetex
    \usepackage{mathspec}
  \else
    \usepackage{fontspec}
  \fi
  \defaultfontfeatures{Ligatures=TeX,Scale=MatchLowercase}
\fi
% use upquote if available, for straight quotes in verbatim environments
\IfFileExists{upquote.sty}{\usepackage{upquote}}{}
% use microtype if available
\IfFileExists{microtype.sty}{%
\usepackage{microtype}
\UseMicrotypeSet[protrusion]{basicmath} % disable protrusion for tt fonts
}{}
\usepackage[margin=2.5cm]{geometry}
\usepackage{hyperref}
\hypersetup{unicode=true,
            pdftitle={Project 1: regression 1},
            pdfauthor={Urvan Christen, Amandine Goffeney, Joseph Vermeil, Lucile Vigué},
            pdfborder={0 0 0},
            breaklinks=true}
\urlstyle{same}  % don't use monospace font for urls
\usepackage{graphicx,grffile}
\makeatletter
\def\maxwidth{\ifdim\Gin@nat@width>\linewidth\linewidth\else\Gin@nat@width\fi}
\def\maxheight{\ifdim\Gin@nat@height>\textheight\textheight\else\Gin@nat@height\fi}
\makeatother
% Scale images if necessary, so that they will not overflow the page
% margins by default, and it is still possible to overwrite the defaults
% using explicit options in \includegraphics[width, height, ...]{}
\setkeys{Gin}{width=\maxwidth,height=\maxheight,keepaspectratio}
\IfFileExists{parskip.sty}{%
\usepackage{parskip}
}{% else
\setlength{\parindent}{0pt}
\setlength{\parskip}{6pt plus 2pt minus 1pt}
}
\setlength{\emergencystretch}{3em}  % prevent overfull lines
\providecommand{\tightlist}{%
  \setlength{\itemsep}{0pt}\setlength{\parskip}{0pt}}
\setcounter{secnumdepth}{0}
% Redefines (sub)paragraphs to behave more like sections
\ifx\paragraph\undefined\else
\let\oldparagraph\paragraph
\renewcommand{\paragraph}[1]{\oldparagraph{#1}\mbox{}}
\fi
\ifx\subparagraph\undefined\else
\let\oldsubparagraph\subparagraph
\renewcommand{\subparagraph}[1]{\oldsubparagraph{#1}\mbox{}}
\fi

%%% Use protect on footnotes to avoid problems with footnotes in titles
\let\rmarkdownfootnote\footnote%
\def\footnote{\protect\rmarkdownfootnote}

%%% Change title format to be more compact
\usepackage{titling}

% Create subtitle command for use in maketitle
\newcommand{\subtitle}[1]{
  \posttitle{
    \begin{center}\large#1\end{center}
    }
}

\setlength{\droptitle}{-2em}

  \title{Project 1: regression 1}
    \pretitle{\vspace{\droptitle}\centering\huge}
  \posttitle{\par}
    \author{Urvan Christen, Amandine Goffeney, Joseph Vermeil, Lucile Vigué}
    \preauthor{\centering\large\emph}
  \postauthor{\par}
      \predate{\centering\large\emph}
  \postdate{\par}
    \date{March 20, 2019}

\usepackage{fancyhdr}
\pagestyle{fancy}
\fancyhead[LE,RO]{Christen, Goffeney, Vermeil, Vigué}

\begin{document}
\maketitle

\begin{verbatim}
## Warning: package 'tidyverse' was built under R version 3.5.3
\end{verbatim}

\begin{verbatim}
## Warning: package 'tidyr' was built under R version 3.5.3
\end{verbatim}

\begin{verbatim}
## Warning: package 'readr' was built under R version 3.5.3
\end{verbatim}

\begin{verbatim}
## Warning: package 'purrr' was built under R version 3.5.3
\end{verbatim}

\begin{verbatim}
## Warning: package 'dplyr' was built under R version 3.5.3
\end{verbatim}

\begin{verbatim}
## Warning: package 'forcats' was built under R version 3.5.3
\end{verbatim}

\begin{verbatim}
## Warning: package 'MASS' was built under R version 3.5.3
\end{verbatim}

\begin{verbatim}
## Warning: package 'GGally' was built under R version 3.5.3
\end{verbatim}

\begin{verbatim}
## Warning: package 'car' was built under R version 3.5.3
\end{verbatim}

\section{Introduction}\label{introduction}

Mercury is a metal present in the environment whose harmful effects on
human health are well assessed (Park and Zheng 2012). In a study led in
2000 (Al-Majed and Preston 2000), scientists collected data on total
mercury and methyl mercury levels in the hair of 100 fishermen of
Kuwait, aged 16 to 58 years, comparing them to those of a control
population of 35 non-fishermen, aged 26 to 35 years. The aim of this
report is to analyse the factors influencing the levels of mercury in
both populations. For the sake of simplicity, we will only focus on
total Hg, leaving out methyl mercury, as both variables are strongly
correlated.

The dataset gathers information about six numerical variables (age,
height, weight, number of fish meals per week and residence time in
Kuwait) and two categorical ones (being a fisherman or not, fish
consumption habits). There is no additional information about gender as
all the participants in the study are males.

A first insight at the data shows that the fishermen population exhibits
higher levels of mercury in their hair. The significance of the
difference between the means of both distributions is assessed by a
\emph{Welch Two Sample t-test} with the alternative hypothesis that the
fishermen population has a average greater level of mercury than the
control population (p-value = 7.473e-05).

\includegraphics{Report_files/figure-latex/unnamed-chunk-3-1.pdf}

\section*{References}\label{references}
\addcontentsline{toc}{section}{References}

\hypertarget{refs}{}
\hypertarget{ref-al2000factors}{}
Al-Majed, NB, and MR Preston. 2000. ``Factors Influencing the Total
Mercury and Methyl Mercury in the Hair of the Fishermen of Kuwait.''
\emph{Environmental Pollution} 109 (2). Elsevier: 239--50.

\hypertarget{ref-park2012human}{}
Park, Jung-Duck, and Wei Zheng. 2012. ``Human Exposure and Health
Effects of Inorganic and Elemental Mercury.'' \emph{Journal of
Preventive Medicine and Public Health} 45 (6). Korean Society for
Preventive Medicine: 344.


\end{document}
