\documentclass[12pt,]{article}
\usepackage{lmodern}
\usepackage{amssymb,amsmath}
\usepackage{ifxetex,ifluatex}
\usepackage{fixltx2e} % provides \textsubscript
\ifnum 0\ifxetex 1\fi\ifluatex 1\fi=0 % if pdftex
  \usepackage[T1]{fontenc}
  \usepackage[utf8]{inputenc}
\else % if luatex or xelatex
  \ifxetex
    \usepackage{mathspec}
  \else
    \usepackage{fontspec}
  \fi
  \defaultfontfeatures{Ligatures=TeX,Scale=MatchLowercase}
\fi
% use upquote if available, for straight quotes in verbatim environments
\IfFileExists{upquote.sty}{\usepackage{upquote}}{}
% use microtype if available
\IfFileExists{microtype.sty}{%
\usepackage{microtype}
\UseMicrotypeSet[protrusion]{basicmath} % disable protrusion for tt fonts
}{}
\usepackage[margin=2.5cm]{geometry}
\usepackage{hyperref}
\hypersetup{unicode=true,
            pdftitle={Project 1: regression 1},
            pdfauthor={Urvan Christen, Amandine Goffeney, Joseph Vermeil, Lucile Vigué},
            pdfborder={0 0 0},
            breaklinks=true}
\urlstyle{same}  % don't use monospace font for urls
\usepackage{graphicx,grffile}
\makeatletter
\def\maxwidth{\ifdim\Gin@nat@width>\linewidth\linewidth\else\Gin@nat@width\fi}
\def\maxheight{\ifdim\Gin@nat@height>\textheight\textheight\else\Gin@nat@height\fi}
\makeatother
% Scale images if necessary, so that they will not overflow the page
% margins by default, and it is still possible to overwrite the defaults
% using explicit options in \includegraphics[width, height, ...]{}
\setkeys{Gin}{width=\maxwidth,height=\maxheight,keepaspectratio}
\IfFileExists{parskip.sty}{%
\usepackage{parskip}
}{% else
\setlength{\parindent}{0pt}
\setlength{\parskip}{6pt plus 2pt minus 1pt}
}
\setlength{\emergencystretch}{3em}  % prevent overfull lines
\providecommand{\tightlist}{%
  \setlength{\itemsep}{0pt}\setlength{\parskip}{0pt}}
\setcounter{secnumdepth}{0}
% Redefines (sub)paragraphs to behave more like sections
\ifx\paragraph\undefined\else
\let\oldparagraph\paragraph
\renewcommand{\paragraph}[1]{\oldparagraph{#1}\mbox{}}
\fi
\ifx\subparagraph\undefined\else
\let\oldsubparagraph\subparagraph
\renewcommand{\subparagraph}[1]{\oldsubparagraph{#1}\mbox{}}
\fi

%%% Use protect on footnotes to avoid problems with footnotes in titles
\let\rmarkdownfootnote\footnote%
\def\footnote{\protect\rmarkdownfootnote}

%%% Change title format to be more compact
\usepackage{titling}

% Create subtitle command for use in maketitle
\providecommand{\subtitle}[1]{
  \posttitle{
    \begin{center}\large#1\end{center}
    }
}

\setlength{\droptitle}{-2em}

  \title{Project 1: regression 1}
    \pretitle{\vspace{\droptitle}\centering\huge}
  \posttitle{\par}
    \author{Urvan Christen, Amandine Goffeney, Joseph Vermeil, Lucile Vigué}
    \preauthor{\centering\large\emph}
  \postauthor{\par}
      \predate{\centering\large\emph}
  \postdate{\par}
    \date{March 20, 2019}

\usepackage{fancyhdr}
\usepackage{wrapfig}
\pagestyle{fancy}
\fancyhead[LE,RO]{Christen, Goffeney, Vermeil, Vigué}

\begin{document}
\maketitle

\section{Introduction}\label{introduction}

Mercury is a metal present in the environment whose harmful effects on
human health are well assessed (Park and Zheng 2012). In a study led in
2000 (Al-Majed and Preston 2000), scientists collected data on total
mercury and methyl mercury levels in the hair of 100 fishermen of
Kuwait, aged 16 to 58 years, comparing them to those of a control
population of 35 non-fishermen, aged 26 to 35 years. The aim of this
report is to analyse the factors influencing the levels of mercury in
both populations. For the sake of simplicity, we will only focus on
total Hg, leaving out methyl mercury, as both variables are strongly
correlated.

\begin{wrapfigure}{R}{0.35\textwidth}

\hfill{}\includegraphics{Report_files/figure-latex/unnamed-chunk-4-1} 

\caption{Boxplots of Hg levels}\label{fig:unnamed-chunk-4}
\end{wrapfigure}

The dataset gathers information about six numerical variables (age,
height, weight, number of fish meals per week and residence time in
Kuwait) and two categorical ones (being a fisherman or not, fish
consumption habits). There is no additional information about gender as
all the participants in the study are males.

A first insight at the data shows that the fishermen population exhibits
higher levels of mercury in their hair. The significance of the
difference between the means of both distributions is assessed by a
\emph{Welch Two Sample t-test} with the alternative hypothesis that the
fishermen population has a average greater level of mercury than the
control population (\emph{p}-value = 7.473e-05).

\section{Exploratory analysis
(Amandine)}\label{exploratory-analysis-amandine}

\paragraph{Overview of the data}\label{overview-of-the-data}

Before fitting a model, let's have a look at the data. The following
table shows the 8 possible values of fishmlwk according to the 2
possible values of fisherman.

\begin{verbatim}
##    
##      0  1  2  3  4  7 14 21
##   0 10 14 11  0  0  0  0  0
##   1  0  0  0  2 12 70  5 11
\end{verbatim}

First, let's notice that for some values, we have a very few people, for
instance only 2 people eat fish 3 times a week. We see also that the
data is really unbalanced, because the fishermen and non fishermen are
separated on the variable ``fishmlwk'': the non fishermen don't eat fish
more than 2 times a week, whereas the fishermen don't eat fish less than
3 times a week. Besides, we face the same unbalanced pattern for the
variables \emph{age} and \emph{restime}.

\paragraph{Multicolinearity}\label{multicolinearity}

To check whether we have multicolinear variables, we use the VIF
function. If all the variables have a result below 5, we keep them.

\begin{verbatim}
##               GVIF Df GVIF^(1/(2*Df))
## fisherman 2.176756  1        1.475383
## age       1.609314  1        1.268587
## restime   1.609630  1        1.268712
## height    1.138614  1        1.067059
## weight    1.220641  1        1.104826
## fishmlwk  1.754183  1        1.324456
## fishpart  1.745741  3        1.097312
\end{verbatim}

Indeed, we observe that all the variables have a VIF below 2, so we
don't eliminate any variable, yet.

\section{Model selection (Joseph)}\label{model-selection-joseph}

Now we are going to use and compare the different methods of model
selection to select the more relevant parameters to explain the TotHg
variations within the population.

\subsection{Stepwise selection}\label{stepwise-selection}

The first attempt is a stepwise selection based on a formula with all
the parameters except for \emph{fisherman}. Indeed, the preliminary
exploration has shown that the \emph{fisherman} variable have
considerable impact on many of the other variables. Therefore to
understand what is the reason behind the TotHg difference between the 2
groups, it seems natural to start with a model without the
\emph{fisherman} variable.

\begin{verbatim}
## 
## Call:
## lm(formula = TotHg ~ weight + fishmlwk, data = dataset)
## 
## Residuals:
##     Min      1Q  Median      3Q     Max 
## -4.8344 -1.3096 -0.2953  0.6279 11.8572 
## 
## Coefficients:
##              Estimate Std. Error t value Pr(>|t|)    
## (Intercept) -10.07682    2.44481  -4.122 6.60e-05 ***
## weight        0.17518    0.03322   5.273 5.34e-07 ***
## fishmlwk      0.15884    0.04175   3.805 0.000216 ***
## ---
## Signif. codes:  0 '***' 0.001 '**' 0.01 '*' 0.05 '.' 0.1 ' ' 1
## 
## Residual standard error: 2.564 on 132 degrees of freedom
## Multiple R-squared:  0.2498, Adjusted R-squared:  0.2384 
## F-statistic: 21.97 on 2 and 132 DF,  p-value: 5.787e-09
\end{verbatim}

With stepwise selection, the selected model is : TotHg \textasciitilde{}
weight + fishmlwk. The intercept and the two coefficients are very
significant. Moreover the signs of the coefficients are not absurd:
while it is not really intuitive that the weight coefficient should be
positive or negative, the coefficient of \emph{fishmlwk} had to be
positive, and it's the case here.

Now to determine possible differences between the fishermen and
non-fishermen, a model based on the interactions between the
\emph{fisherman} variable and all the others is proposed. This could
show if a certain variable is very relevant concerning fishermen but
less when it comes to non-fishermen, for instance.

\begin{verbatim}
## 
## Call:
## lm(formula = TotHg ~ fisherman:weight + fisherman:fishmlwk, data = dataset)
## 
## Residuals:
##     Min      1Q  Median      3Q     Max 
## -5.0705 -1.1391 -0.2026  0.6525 11.4266 
## 
## Coefficients:
##                     Estimate Std. Error t value Pr(>|t|)    
## (Intercept)         -9.26610    2.48392  -3.730 0.000284 ***
## fisherman0:weight    0.14464    0.03671   3.939 0.000133 ***
## fisherman1:weight    0.17361    0.03419   5.078 1.29e-06 ***
## fisherman0:fishmlwk  1.09393    0.60324   1.813 0.072076 .  
## fisherman1:fishmlwk  0.09691    0.05274   1.837 0.068420 .  
## ---
## Signif. codes:  0 '***' 0.001 '**' 0.01 '*' 0.05 '.' 0.1 ' ' 1
## 
## Residual standard error: 2.532 on 130 degrees of freedom
## Multiple R-squared:  0.2797, Adjusted R-squared:  0.2575 
## F-statistic: 12.62 on 4 and 130 DF,  p-value: 1.053e-08
\end{verbatim}

The best model is now : TotHg \textasciitilde{} fisherman:weight +
fisherman:fishmlwk, ie the interactions between fisherman and the
variables of the previous model. While the coefficients for
fisherman0:weight and fisherman1:weight stays very significant,
fisherman0:fishmlwk and fisherman1:fishmlwk are barely significant (but
their p-value is very close to 0.05) ; this comes from the fact that
within each of the subpopulation \{fishermen\} and \{non-fishermen\} the
range in which fishmlwk varies is very reduced compared to the whole
population (see the Exploratory analysis). Moreover, when looking at the
values of the coefficients, it appears that for the non-fishermen, the
number of fish meal per week has a very preponderant effect, while among
fishermen, the contribution of weight and fish meal per week are almost
even.

In order to check the dependency of the selected model to the selection
technique, we have tried backward and forward selection and obtained
similar models.

\section{Results and discussion
(Urvan)}\label{results-and-discussion-urvan}

Fit model (* fishermen) (I picked the model given by stepwise selection)

\begin{verbatim}
## 
## Call:
## lm(formula = hg.form.selected, data = dataset)
## 
## Residuals:
##     Min      1Q  Median      3Q     Max 
## -5.2935 -1.1455 -0.1474  0.5763 11.2698 
## 
## Coefficients:
##                      Estimate Std. Error t value Pr(>|t|)    
## fisherman0           -1.41565    6.93854  -0.204 0.838654    
## fisherman1          -10.41533    2.65475  -3.923 0.000141 ***
## fisherman0:weight     0.03313    0.09907   0.334 0.738597    
## fisherman1:weight     0.18909    0.03644   5.189    8e-07 ***
## fisherman0:fishmlwk   1.53107    0.70201   2.181 0.030997 *  
## fisherman1:fishmlwk   0.09829    0.05266   1.867 0.064235 .  
## ---
## Signif. codes:  0 '***' 0.001 '**' 0.01 '*' 0.05 '.' 0.1 ' ' 1
## 
## Residual standard error: 2.528 on 129 degrees of freedom
## Multiple R-squared:  0.7325, Adjusted R-squared:  0.7201 
## F-statistic: 58.89 on 6 and 129 DF,  p-value: < 2.2e-16
\end{verbatim}

This regression shows several intersting results:

First, we get two coefficients very significantly different from 0
(\emph{p} \textless{} 0.001), both concerning fisherman population. The
fact that there are less significant coefficients for non-fisherman
population could be explained by several causes:

\begin{itemize}
\tightlist
\item
  The non-fisherman population could be too small to properly show the
  size effect of those observables.
\item
  The non-fisherman population may not have settled long enough in the
  place to have repercussion on the mercury levels.
\end{itemize}

However, those are only suppositions in order to explain the
distribution of the \emph{p}-values, but none of them have been proven.

On the other side, the number of meals composed of fish
(\emph{fishmlwk}) seems to be contributing to the mercury levels of both
populations (\emph{p}-values of 0.03 for non-fisherman and 0.06 for
fisherman). However, we can see that there is a huge difference between
the two contributions of a factor 16.

However, the most significant coefficient is for \emph{weight} for
fisherman population, with a \emph{p}-value of 8e-07.

\includegraphics{Report_files/figure-latex/unnamed-chunk-16-1.pdf}
\includegraphics{Report_files/figure-latex/unnamed-chunk-16-2.pdf}
\includegraphics{Report_files/figure-latex/unnamed-chunk-16-3.pdf}
\includegraphics{Report_files/figure-latex/unnamed-chunk-16-4.pdf}
\includegraphics{Report_files/figure-latex/unnamed-chunk-16-5.pdf}
\includegraphics{Report_files/figure-latex/unnamed-chunk-16-6.pdf}
\includegraphics{Report_files/figure-latex/unnamed-chunk-16-7.pdf}

\section{Conclusion (Lucile)}\label{conclusion-lucile}

3 lignes

Les principales variables explicatives sont:

\begin{itemize}
\tightlist
\item
  La consommation de poisson =\textgreater{} logique
\item
  Le poids =\textgreater{} plus étonnant mais on sait que les cellules
  graisseuses stockent les toxiques
\end{itemize}

\section*{References}\label{references}
\addcontentsline{toc}{section}{References}

\hypertarget{refs}{}
\hypertarget{ref-al2000factors}{}
Al-Majed, NB, and MR Preston. 2000. ``Factors Influencing the Total
Mercury and Methyl Mercury in the Hair of the Fishermen of Kuwait.''
\emph{Environmental Pollution} 109 (2). Elsevier: 239--50.

\hypertarget{ref-park2012human}{}
Park, Jung-Duck, and Wei Zheng. 2012. ``Human Exposure and Health
Effects of Inorganic and Elemental Mercury.'' \emph{Journal of
Preventive Medicine and Public Health} 45 (6). Korean Society for
Preventive Medicine: 344.


\end{document}
