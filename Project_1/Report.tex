\documentclass[12pt,]{article}
\usepackage{lmodern}
\usepackage{amssymb,amsmath}
\usepackage{ifxetex,ifluatex}
\usepackage{fixltx2e} % provides \textsubscript
\ifnum 0\ifxetex 1\fi\ifluatex 1\fi=0 % if pdftex
  \usepackage[T1]{fontenc}
  \usepackage[utf8]{inputenc}
\else % if luatex or xelatex
  \ifxetex
    \usepackage{mathspec}
  \else
    \usepackage{fontspec}
  \fi
  \defaultfontfeatures{Ligatures=TeX,Scale=MatchLowercase}
\fi
% use upquote if available, for straight quotes in verbatim environments
\IfFileExists{upquote.sty}{\usepackage{upquote}}{}
% use microtype if available
\IfFileExists{microtype.sty}{%
\usepackage{microtype}
\UseMicrotypeSet[protrusion]{basicmath} % disable protrusion for tt fonts
}{}
\usepackage[margin=2.5cm]{geometry}
\usepackage{hyperref}
\hypersetup{unicode=true,
            pdftitle={Factors influencing the levels of mercury in the hair of fishermen and non-fishermen},
            pdfauthor={Urvan Christen, Amandine Goffeney, Joseph Vermeil, Lucile Vigué},
            pdfborder={0 0 0},
            breaklinks=true}
\urlstyle{same}  % don't use monospace font for urls
\usepackage{graphicx,grffile}
\makeatletter
\def\maxwidth{\ifdim\Gin@nat@width>\linewidth\linewidth\else\Gin@nat@width\fi}
\def\maxheight{\ifdim\Gin@nat@height>\textheight\textheight\else\Gin@nat@height\fi}
\makeatother
% Scale images if necessary, so that they will not overflow the page
% margins by default, and it is still possible to overwrite the defaults
% using explicit options in \includegraphics[width, height, ...]{}
\setkeys{Gin}{width=\maxwidth,height=\maxheight,keepaspectratio}
\IfFileExists{parskip.sty}{%
\usepackage{parskip}
}{% else
\setlength{\parindent}{0pt}
\setlength{\parskip}{6pt plus 2pt minus 1pt}
}
\setlength{\emergencystretch}{3em}  % prevent overfull lines
\providecommand{\tightlist}{%
  \setlength{\itemsep}{0pt}\setlength{\parskip}{0pt}}
\setcounter{secnumdepth}{0}
% Redefines (sub)paragraphs to behave more like sections
\ifx\paragraph\undefined\else
\let\oldparagraph\paragraph
\renewcommand{\paragraph}[1]{\oldparagraph{#1}\mbox{}}
\fi
\ifx\subparagraph\undefined\else
\let\oldsubparagraph\subparagraph
\renewcommand{\subparagraph}[1]{\oldsubparagraph{#1}\mbox{}}
\fi

%%% Use protect on footnotes to avoid problems with footnotes in titles
\let\rmarkdownfootnote\footnote%
\def\footnote{\protect\rmarkdownfootnote}

%%% Change title format to be more compact
\usepackage{titling}

% Create subtitle command for use in maketitle
\providecommand{\subtitle}[1]{
  \posttitle{
    \begin{center}\large#1\end{center}
    }
}

\setlength{\droptitle}{-2em}

  \title{Factors influencing the levels of mercury in the hair of fishermen and
non-fishermen}
    \pretitle{\vspace{\droptitle}\centering\huge}
  \posttitle{\par}
    \author{Urvan Christen, Amandine Goffeney, Joseph Vermeil, Lucile Vigué}
    \preauthor{\centering\large\emph}
  \postauthor{\par}
    \date{}
    \predate{}\postdate{}
  
\usepackage{booktabs}
\usepackage{longtable}
\usepackage{array}
\usepackage{multirow}
\usepackage{wrapfig}
\usepackage{float}
\usepackage{colortbl}
\usepackage{pdflscape}
\usepackage{tabu}
\usepackage{threeparttable}
\usepackage{threeparttablex}
\usepackage[normalem]{ulem}
\usepackage{makecell}
\usepackage{xcolor}

\usepackage{fancyhdr}
\usepackage{wrapfig}
\pagestyle{fancy}
\fancyhead[LE,RO]{Christen, Goffeney, Vermeil, Vigué}

\begin{document}
\maketitle

\subsection{Introduction}\label{introduction}

Mercury is a metal present in the environment whose harmful effects on
human health are well known (Park and Zheng 2012). In (Al-Majed and
Preston 2000), total mercury and methyl mercury levels in the hair of
100 fishermen of Kuwait, aged 16 to 58 years, were compared to those of
a control population of 35 non-fishermen, aged 26 to 35 years. The aim
of our study is to analyse the factors influencing the levels of mercury
in both populations. For the sake of simplicity, we will only focus on
total Hg, leaving out methyl mercury, since both variables are strongly
correlated (shown in the paper). The dataset contains six numerical
variables (age, height, weight, number of fish meals per week and
residence time in Kuwait) and two categorical variables (being a
fisherman or not, fish consumption habits). All study participants are
male.

\section{Exploratory analysis}\label{exploratory-analysis}

Before fitting a model, we first look at the data. Table
\ref{tbl:fishmlwk} shows the distribution of individuals according to
the number of fish meals per week and the two groups. We note that for
some values, there are very few people. For instance only two people eat
fish three times per week. We also see that the number of fish meals per
week is completely separable by population group. This explains the
strong correlation observed between these two variables in the
correlation matrix (Table \ref{tbl:correlation}). Besides, we also
observe that the variables age and residence time in Kuwait are quite
correlated with being a fisherman or not. To check whether we have
multicolinear variables, we use the variance inflation factor (VIF). We
set the following criteria: we keep only the variables that have a
result below 5. We observe that all the variables have a variance
inflation factor below 2, so we do not eliminate any variable, for the
moment.

\begin{table}[t]

\caption{\label{tab:unnamed-chunk-6}\label{tbl:fishmlwk}Distribution of the number of fish meals accross fishermen and non-fishermen populations}
\centering
\begin{tabular}{lrrrrrrrr}
\toprule
  & 0 & 1 & 2 & 3 & 4 & 7 & 14 & 21\\
\midrule
\rowcolor{gray!6}  non-fisherman & 10 & 14 & 11 & 0 & 0 & 0 & 0 & 0\\
fisherman & 0 & 0 & 0 & 2 & 12 & 70 & 5 & 11\\
\bottomrule
\end{tabular}
\end{table}

\begin{table}[t]

\caption{\label{tab:unnamed-chunk-7}\label{tbl:correlation}Correlation matrix}
\centering
\begin{tabular}{l|r|r|r|r|r|r|r|r}
\hline
  & fisherman & age & restime & height & weight & fishmlwk & fishpart & TotHg\\
\hline
\rowcolor{gray!6}  fisherman & 1.00 & 0.25 & 0.25 & -0.06 & -0.09 & 0.61 & 0.46 & 0.23\\
\hline
age & 0.25 & 1.00 & 0.58 & 0.00 & 0.05 & 0.26 & -0.01 & 0.16\\
\hline
\rowcolor{gray!6}  restime & 0.25 & 0.58 & 1.00 & -0.05 & 0.11 & 0.19 & 0.00 & 0.06\\
\hline
height & -0.06 & 0.00 & -0.05 & 1.00 & 0.30 & -0.04 & -0.03 & 0.19\\
\hline
\rowcolor{gray!6}  weight & -0.09 & 0.05 & 0.11 & 0.30 & 1.00 & 0.04 & -0.05 & 0.41\\
\hline
fishmlwk & 0.61 & 0.26 & 0.19 & -0.04 & 0.04 & 1.00 & 0.19 & 0.30\\
\hline
\rowcolor{gray!6}  fishpart & 0.46 & -0.01 & 0.00 & -0.03 & -0.05 & 0.19 & 1.00 & 0.11\\
\hline
TotHg & 0.23 & 0.16 & 0.06 & 0.19 & 0.41 & 0.30 & 0.11 & 1.00\\
\hline
\end{tabular}
\end{table}

\begin{figure}
\centering
\includegraphics{Report_files/figure-latex/unnamed-chunk-8-1.pdf}
\caption{\label{fig:pair_plots}Pair plots}
\end{figure}

\subsection{Model selection}\label{model-selection}

We now use the different methods of model selection to select the more
relevant variables to explain the \emph{TotHg} variations within the
population. We first apply a stepwise selection based on a formula with
all the variables. With stepwise selection, the selected model is :
\(TotHg = \beta_0 + \beta_1 \cdot fisherman + \beta_2 \cdot age + \beta_3 \cdot restime + \beta_4 \cdot weight + \beta_5 \cdot fishmlwk\).
The intercept and the two coefficients are highly significant. Moreover
the signs of the coefficients are not absurd: while it is not really
intuitive that the weight coefficient should be positive or negative,
the coefficient of \emph{fishmlwk} has to be positive, and it is the
case here. Now, to determine possible differences between the fishermen
and non-fishermen, a model based on the interactions between the
\emph{fisherman} variable and all the others is proposed. The best model
is now
\(TotHg = \beta_0 + \beta_1 \cdot fisherman + \beta_2 \cdot fisherman \cdot weight + \beta_3 \cdot fisherman \cdot fishmlwk\).
The coefficients for \emph{fisherman}, \emph{fisherman:weight} and the
intercept are very significant. On the contrary,
\emph{fisherman:fishmlwk} is barely significant, this may be due to the
fact that both variables are highly correlated as explained before.

TOUT LE PARAGRAPHE QUI SUIT NE FAIT PLUS SENS MANTENANT Moreover, when
looking at the values of the coefficients, it appears that for the
non-fishermen, the number of fish meal per week has a very preponderant
effect compared to the weight influence, while among fishermen, the
contribution of \emph{weight} is twice the one of \emph{fishmlwk}.
Eventually the value of \(\beta_1\) (corresponding to \emph{fisherman1}
variable) is surprising: it implies that the fact of being a fisherman
gives you -8.99 mg/g Hg compared to non-fishermen. Yet this comes from
the increased value of fisherman1:weight, which will make the overall Hg
concentration more important among fishermen, as expected. In order to
check the dependency of the selected model to the selection technique,
we have applied backward and forward selection and obtained similar
models.

\begin{table}[t]

\caption{\label{tab:unnamed-chunk-12}Full model regression results}
\centering
\begin{tabular}{l|l|l|l|l}
\hline
  & Estimate & Std. Error & t value & Pr(>|t|)\\
\hline
\rowcolor{gray!6}  (Intercept) & -12.68 & 2.71 & -4.68 & 7.0e-06\\
\hline
fisherman & 1.11 & 0.65 & 1.70 & 9.1e-02\\
\hline
\rowcolor{gray!6}  age & 0.05 & 0.03 & 1.43 & 1.6e-01\\
\hline
restime & -0.08 & 0.05 & -1.45 & 1.5e-01\\
\hline
\rowcolor{gray!6}  weight & 0.19 & 0.03 & 5.58 & 1.4e-07\\
\hline
fishmlwk & 0.10 & 0.05 & 1.82 & 7.2e-02\\
\hline
\end{tabular}
\end{table}

\begin{table}[t]

\caption{\label{tab:unnamed-chunk-14}Model regression results}
\centering
\begin{tabular}{l|l|l|l|l}
\hline
  & Estimate & Std. Error & t value & Pr(>|t|)\\
\hline
\rowcolor{gray!6}  (Intercept) & 2.62 & 0.44 & 5.96 & 2.2e-08\\
\hline
fisherman & -13.03 & 2.76 & -4.72 & 6.1e-06\\
\hline
\rowcolor{gray!6}  fisherman:weight & 0.19 & 0.04 & 5.05 & 1.5e-06\\
\hline
fisherman:fishmlwk & 0.10 & 0.05 & 1.82 & 7.2e-02\\
\hline
\end{tabular}
\end{table}

\subsection{Results and discussion}\label{results-and-discussion}

We now turn to discussing the results of the model:
\(\beta_0 + \beta_1 \cdot fisherman + \beta_2 \cdot fisherman \cdot weight + \beta_3 \cdot fisherman \cdot fishmlwk\).

\subsubsection{Difference between fisherman and control
populations}\label{difference-between-fisherman-and-control-populations}

First, we get two coefficients very significantly different from 0
(\emph{p} \textless{} 0.001), both concerning fisherman population. The
fact that there are less significant coefficients for non-fisherman
population could be explained by several causes:

\begin{itemize}
\tightlist
\item
  The non-fisherman population could be too small to properly show the
  size effect of those observables.
\item
  The non-fisherman population may not have settled long enough in the
  place to have repercussion on the mercury levels.
\end{itemize}

However, those are only suppositions in order to explain the
distribution of the \emph{p}-values, but none of them have been proven.
Further experiments are needed in order to show whether or not those
observables have a really different effect on both populations.

\subsubsection{Number of fish meals per
week}\label{number-of-fish-meals-per-week}

TOUTE CETTE PARTIE DOIT ETRE REVUE On the other side, the number of
meals composed of fish (\emph{fishmlwk}) seems to be contributing to the
mercury levels of both populations (\emph{p}-values of XXXXXXX for
non-fisherman and XXXXXX for fisherman). However, we can see that there
is a huge difference between the two contributions of a factor XXXXX.

Here again, we can come up with some possible explanations, needing
further inquiries:

\begin{itemize}
\item
  The distribution of \emph{fishmlwk} is very different between both
  populations and thus may lead to different coefficients if the effect
  of this observables is not truly linear (\emph{e.g.} a logarithmic
  effect that could take some ceiling effects into account, \emph{i.e.}
  the fact that past a certain dose, the hair cannot absorb more
  mercury).
\item
  The observable does not reflect entirely the quantity of fish eaten,
  since one can eat more or less fish per meal. The weight of fish eaten
  per week, might be a more accurate observable to study.
\end{itemize}

Here again, more experiments are required to confirm or reject those
hypotheses.

\subsubsection{Weight}\label{weight}

However, the most significant coefficient is for \emph{weight} for
fisherman population, with a \emph{p}-value of 7e-02. This coefficient
suggests a high positive correlation between the weight of the fisherman
and the concentration of mercury in its hair.

The fact that the weight has a positive influence on this concentration
was unexpected, since a concentration and not an absolute quantity was
measured.

However, even though it was unexpected it has many possible explanations
such as the fact that weight is much likely correlated with adiposity
more susceptible to catch toxins than other tissues. Another explanation
could be that the fatter, the more one eats and possibly ingests mercury
that could fix in the hair; since hair weight isn't likely to be
correlated with body weight, it could explain the high mercury
concentration in hair.

Here again, further experiments are needed in order to support or reject
those hypotheses.

\subsubsection{Diagnostic plots}\label{diagnostic-plots}

\includegraphics{Report_files/figure-latex/unnamed-chunk-18-1.pdf}
\includegraphics{Report_files/figure-latex/unnamed-chunk-18-2.pdf}

The model does not seem to be really homoscedastic. Variance is much
higher for fishermen than for non-fishermen. However, within each class,
the variance is overall well distributed, even if it tends to be a
little more spread for high fitted values.

We have a heavy tailed distribution of residuals, with a very heavy
right tail. It could be explained by a non-linear relation between the
variables and the concentration of mercury.

\subsection{Conclusion}\label{conclusion}

We have built a simple model that can help to explain the levels of
mercury observed in a fishermen population compared to a control group.
It appears that the variables that have the most significant influence
over the measured levels of mercury are the weight of the individual and
the frequency at which they eat fish. The former can seem surprising
even though some hypotheses can be formed to account for the influence
of weight on mercury levels. The latter may be the main explanation for
the differences observed between our two groups: fishermen eat fish much
more often than non-fishermen, since fish is a well-known source of
mercury it seems logical to see a positive correlation between fish meal
frequency and mercury levels and thus to observe higher mercury levels
in fishermen populations compared to non-fishermen.

\subsection*{References}\label{references}
\addcontentsline{toc}{subsection}{References}

\hypertarget{refs}{}
\hypertarget{ref-al2000factors}{}
Al-Majed, NB, and MR Preston. 2000. ``Factors Influencing the Total
Mercury and Methyl Mercury in the Hair of the Fishermen of Kuwait.''
\emph{Environmental Pollution} 109 (2). Elsevier: 239--50.

\hypertarget{ref-park2012human}{}
Park, Jung-Duck, and Wei Zheng. 2012. ``Human Exposure and Health
Effects of Inorganic and Elemental Mercury.'' \emph{Journal of
Preventive Medicine and Public Health} 45 (6). Korean Society for
Preventive Medicine: 344.


\end{document}
